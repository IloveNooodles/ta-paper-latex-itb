\section{Contoh Subbab}
Perujukan literatur dapat dilakukan dengan menambahkan entri baru di berkas. Tulisan ini merujuk pada \parencite{vasp1} atau \parencite{codingame2021media} dan \parencite{4026885}

Sekarang mau ke bab berapa yaaaa.... hmm... ke bab \ref{chapter:studi-literatur} ahhhhh.

\blindtext

\subsection{Contoh Subsubbab}
\label{subsec:contoh-subsec}

\blindtext

\begin{figure}[ht]
	\centering
	\includegraphics[width=0.5\textwidth]{resources/cover-ganesha.jpg}
	\caption{Contoh gambar}
\end{figure}

\subsubsection{Subsubsubbab}
\blindtext

\subsubsubsection{sub sub sub sub bab}
\blindtext

\begin{table}[h]
	\caption{Tabel random}
	\vspace{0.25cm}
	\begin{center}
		\begin{tabular}{|c|c|c|c|}
			\hline
			Title1 & Title2 & Title3 & Title4  \tabularnewline
			\hline
			1647   & 1.97   & 0.68   & 1.90 \tabularnewline
			2301   & 2.92   & 1.06   & 2.75 \tabularnewline
			2969   & 3.23   & 1.16   & 3.78 \tabularnewline
			3791   & 4.39   & 1.40   & 4.14 \tabularnewline
			4625   & 6.72   & 1.87   & 5.59 \tabularnewline
			\hline
		\end{tabular}
	\end{center}
\end{table}