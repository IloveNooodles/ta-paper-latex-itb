%--------------------------------------------------------------------%
%
% Berkas utama templat LaTeX.
%
% author Petra Barus, Peb Ruswono Aryan, Faris Rizki Ekananda
%
%--------------------------------------------------------------------%
%
% Berkas ini berisi struktur utama dokumen LaTeX yang akan dibuat.
%
%--------------------------------------------------------------------%

\documentclass[bahasa, 12pt, a4paper, onecolumn, oneside, final]{report}

%-------------------------------------------------------------------%
%
% Konfigurasi dokumen LaTeX untuk laporan tesis IF ITB
%
% @author Petra Novandi
%
%-------------------------------------------------------------------%
%
% Berkas asli berasal dari Steven Lolong
%
%-------------------------------------------------------------------%

% Ukuran kertas
\special{papersize=210mm,297mm}

% Setting margin
\usepackage[top=2cm,bottom=2cm,left=4cm,right=3cm]{geometry}

% Math package
\usepackage{mathptmx}

% 4th Sectioning
\usepackage{titlesec}
\newcommand{\subsubsubsection}[1]{\paragraph{#1}\mbox{}\\}

\titleformat{\subsubsubsection}
{\normalfont\normalsize\bfseries}{\theparagraph}{1em}{}
\titlespacing*{\subsubsubsection}
{0pt}{3.25ex plus 1ex minus .2ex}{1.5ex plus .2ex}

% Judul bahasa Indonesia
\usepackage[bahasa]{babel}

% Format citation
\usepackage[utf8]{inputenc}
\usepackage[style=apa,backend=biber]{biblatex}
\usepackage{longtable}
\usepackage{graphicx}
\usepackage{subfig}
\usepackage{titling}
\usepackage{booktabs}
\usepackage{tabularx}
\usepackage{blindtext}
\usepackage{sectsty}
\usepackage{chngcntr}
\usepackage{etoolbox}
\usepackage{array}
\usepackage[hidelinks]{hyperref}       % Package untuk link di daftar isi. Ubah jadi \usepackage[hidelinks]{hyperref} apabila ingin menghilangkan kotak merah disekitar link
\usepackage{titlesec}       % Package Format judul
\usepackage{titletoc}       % Package Format judul di toc
\usepackage{tocbibind}      % Package untuk masukkan toc, lot, lof ke Daftar Isi
\usepackage{scrwfile}       % Package untuk membuat Daftar Lampiran dari toc
\usepackage{parskip}
\usepackage{afterpage}
\usepackage{relsize}
\usepackage{xcolor, colortbl}
\usepackage{setspace}
\usepackage{listings}

\graphicspath{{resources/}}   % letak direktori penyimpanan gambar

% Setting daftar lampiran
\newcommand*{\lopname}{DAFTAR LAMPIRAN}
\TOCclone[\lopname]{toc}{atoc}
\addtocontents{atoc}{\protect\value{tocdepth}=-1}
\newcommand\listofappendices{
	\cleardoublepage
	\phantomsection
	\listofatoc
	\addcontentsline{toc}{chapter}{\lopname}
}

\newcommand*\savedtocdepth{}
\AtBeginDocument{%
	\edef\savedtocdepth{\the\value{tocdepth}}%
}

\let\originalappendix\appendix
\renewcommand\appendix{%
	\originalappendix
	\cleardoublepage
	\addtocontents{toc}{\protect\value{tocdepth}=-1}%
	\addtocontents{atoc}{\protect\value{tocdepth}=\savedtocdepth}%

	\titlecontents{chapter}
	[0pt]
	{\bfseries}
	{Lampiran \thecontentslabel.\quad}
	{}
	{\hfill\contentspage}

	\titleformat{\chapter}[block]
	{\bfseries}
	{\chaptertitlename\ \thechapter.\quad}{0pt}
	{\bfseries}
}

% Hilangkan titik pada toc
\makeatletter
\renewcommand{\@dotsep}{1}
\makeatother

% Setel title pada chapter-chapter di toc, lof, lot
\titlecontents{chapter}
[0pt]
{\bfseries}
{\MakeUppercase{Bab} \thecontentslabel\quad\uppercase}
{}
{\mdseries\titlerule*[0.35em]{.}\bfseries\contentspage}
\titlecontents{figure}
[0pt]
{}
{Gambar \thecontentslabel.\quad}
{}
{\mdseries\titlerule*[0.35em]{.}\bfseries\contentspage}
\titlecontents{table}
[0pt]
{}
{Tabel \thecontentslabel.\quad}
{}
{\mdseries\titlerule*[0.35em]{.}\bfseries\contentspage}

% Masukin Daftar Pustaka ke toc
\let\originalprintbibliography\printbibliography
\renewcommand\printbibliography{%
	\phantomsection
	\cleardoublepage
	\originalprintbibliography
	\addcontentsline{toc}{chapter}{\bibname}
}

% Line satu setengah spasi
\renewcommand{\baselinestretch}{1.5}

% Setting judul
\chapterfont{\centering \large}
\titleformat{\chapter}[display]
{\Large\centering\bfseries}
{\chaptertitlename\ \thechapter}{0pt}
{\Large\bfseries\uppercase}

% Setting nomor pada subbsubsubbab
\setcounter{secnumdepth}{4}

\makeatletter

\makeatother

% Counter untuk figure dan table.
\counterwithin{figure}{chapter}
\counterwithin{table}{chapter}

% Define blank page
\newcommand*{\blankpage}{\afterpage{\null\newpage}}

% Translate autoref into Indonesian
\renewcommand*{\equationautorefname}{Persamaan}%
\renewcommand*{\footnoteautorefname}{catatan kaki}%
\renewcommand*{\itemautorefname}{item}%
\renewcommand*{\figureautorefname}{Gambar}%
\renewcommand*{\tableautorefname}{Tabel}%
\renewcommand*{\partautorefname}{Bagian}%
\renewcommand*{\appendixautorefname}{Lampiran}%
\renewcommand*{\chapterautorefname}{Bab}%
\renewcommand*{\sectionautorefname}{Subbab}%
\renewcommand*{\subsectionautorefname}{Subsubbab}%
\renewcommand*{\subsubsectionautorefname}{Subsubsubbab}%
\renewcommand*{\paragraphautorefname}{paragraf}%
\renewcommand*{\subparagraphautorefname}{subparagraf}%
\renewcommand*{\FancyVerbLineautorefname}{garis}%
\renewcommand*{\theoremautorefname}{Teorema}%
\renewcommand*{\pageautorefname}{halaman}%

% Format to ignore underflow hbadness
\hbadness=99999
%--------------------------------------------------------------------%
%
% Hypenation untuk Bahasa Indonesia
%
% @author Petra Barus
%
%--------------------------------------------------------------------%
%
% Secara otomatis LaTeX dapat langsung memenggal kata dalam dokumen,
% tapi sering kali terdapat kesalahan dalam pemenggalan kata. Untuk
% memperbaiki kesalahan pemenggalan kata tertentu, cara pemenggalan
% kata tersebut dapat ditambahkan pada dokumen ini. Pemenggalan
% dilakukan dengan menambahkan karakter '-' pada suku kata yang
% perlu dipisahkan.
%
% Contoh pemenggalan kata 'analisa' dilakukan dengan 'a-na-li-sa'
%
%--------------------------------------------------------------------%

\hyphenation {
	% A
	%
	a-na-li-sa
	a-pli-ka-si
	a-lo-ka-si
	an-ta-ra
	ada-nya
	akan
	% B
	%
	be-be-ra-pa
	ber-ge-rak
	be-ri-kut
	ber-ko-mu-ni-ka-si
	buah
	Bouvet
	ber-fo-kus
	ber-fung-si
	ber-ja-lan
	% C
	%
	ca-ri
	Carzaniga
	cloud
	CloudFormation
	con-tai-ner
	ClusterIP
	% D
	%
	da-e-rah
	di-nya-ta-kan
	de-fi-ni-si
	di-bu-tuh-kan
	di-gu-na-kan
	di-tam-bah-kan-nya
	di-tem-pat-kan
	di-la-ku-kan
	di-kem-bang-kan
	di-im-ple-men-ta-si-kan
	da-pat
	di-ka-te-go-ri-kan
	de-ngan
	di-kem-bang-kan
	da-ta
	% E
	%
	e-ner-gi
	eks-klu-sif
	eks-ter-nal
	% F
	%
	fa-si-li-tas
	% G
	%
	ga-bung-an
	% H
	%
	ha-lang-an
	ha-sil
	hell
	% I
	% 
	i-nduk
	in-for-ma-si
	im-ple-men-tasi
	% J
	%
	% K
	%
	kom-po-si-si
	ka-re-na
	ke-sa-ba-ran-nya
	ka-me-ra
	kua-li-tas
	ke-na-ngan
	kom-plek-si-tas
	ke-ti-ka
	ke-le-bi-han-nya
	ke-gi-a-tan
	ko-mu-ni-ka-si
	ke-cil
	% L
	%
	la-ya-nan
	% M
	%
	me-ngu-ra-ngi
	meng-eva-lu-a-si
	me-nge-lo-la
	men-da-lam
	men-ja-lan-kan
	mak-si-mal
	me-nye-le-sai-kan
	me-ngunjungi
	men-du-kung
	me-nu-rut
	me-la-ku-kan
	mem-buat
	men-daftar-kan
	me-ngu-sul-kan
	me-mi-li-ki
	meng-gu-na-kan
	men-ja-di
	me-ru-pa-kan
	men-ja-ga
	me-mu-dah-kan
	me-ne-rus-
	mem-pro-ses
	% N
	%
	Na-mun
	% O
	%
	ob-so-lete
	or-kes-tra-si
	oto-ma-ti-sa-si
	% P
	%
	pro-vi-der
	pe-ru-sa-ha-an
	pe-rang-kat
	pro-ses
	plat-form
	pro-duk-si
	pe-ne-li-tian
	pe-ru-ba-han
	pa-ra-dig-ma
	pe-man-tau-an
	pe-ngum-pu-lan
	pa-ckage
	% Q
	%
	quality
	% R
	%
	% S
	se-la-in
	stan-dar-di-sasi
	se-cu-ri-ty
	so-lu-si
	se-lu-ruh
	Soft-ware
	soft-ware
	se-buah
	se-ca-ra
	%
	% T
	% 
	ter-li-bat
	ter-pi-sah-kan
	ter-mi-nal
	ter-ba-tas
	% U
	%
	un-tuk
	% V
	%
	% W
	%
	% X
	%
	% Y
	% 
	% Z
	%
}

%--------------------------------------------------------------------%
%
% Hypenation for English
%
% @author Muhammad Garebaldhie
%
%--------------------------------------------------------------------%
%
% Secara otomatis LaTeX dapat langsung memenggal kata dalam dokumen,
% tapi sering kali terdapat kesalahan dalam pemenggalan kata. Untuk
% memperbaiki kesalahan pemenggalan kata tertentu, cara pemenggalan
% kata tersebut dapat ditambahkan pada dokumen ini. Pemenggalan
% dilakukan dengan menambahkan karakter '-' pada suku kata yang
% perlu dipisahkan.
%
% Contoh pemenggalan kata 'analisa' dilakukan dengan 'a-na-li-sa'
%
%--------------------------------------------------------------------%


\makeatletter

\makeatother

\addbibresource{references.example.bib}

\begin{document}

\title{Judul TA}
\date{}
\author{
	Muhammad Garebaldhie ER Rahman \\
	NIM: 13520029
}
\newcommand\tanggalpengesahan{7 Juli 2024}

\pagenumbering{roman}
\setcounter{page}{1}

\clearpage
\pagestyle{empty}

\begin{center}
	\smallskip

	\Large \bfseries \MakeUppercase{\thetitle}
	\vfill

	\Large Laporan Tugas Akhir
	\vfill

	\large Disusun sebagai syarat kelulusan tingkat sarjana
	\vfill

	\large Oleh

	\Large \theauthor

	\vfill
	\begin{figure}[ht]
		\centering
		\includegraphics[width=0.15\textwidth]{cover-ganesha.jpg}
	\end{figure}
	\vfill

	\large
	\uppercase{
		Program Studi Teknik Informatika \\
		Sekolah Teknik Elektro \& Informatika \\
		Institut Teknologi Bandung
	}

	Juli 2024

\end{center}

\clearpage

\clearpage
\pagestyle{empty}

\begin{center}
	\smallskip

	\Large \bfseries \MakeUppercase{\thetitle}
	\vfill

	\Large Laporan Tugas Akhir
	\vfill

	\large Oleh

	\Large \theauthor

	\large Program Studi Teknik Informatika \\

	\normalsize \normalfont
	Sekolah Teknik Elektro dan Informatika \\
	Institut Teknologi Bandung \\

	\vfill
	\normalsize \normalfont
	Telah disetujui dan disahkan sebagai Laporan Tugas Akhir \\
	% Telah disetujui dan disahkan sebagai Draft Laporan Tugas Akhir \\
	di Bandung, pada tanggal \tanggalpengesahan

	\vspace{0.5cm}
	Pembimbing,

	\vfill
	\underline{Dr. techn. Muhammad Zuhri Catur Candra, S.T, M.T.
	} \\
	NIP. 19770921 201012 1 002

\end{center}
\clearpage

\input{chapters/ta/statement}

\pagestyle{plain}

\clearpage
\chapter*{ABSTRAK}
\addcontentsline{toc}{chapter}{ABSTRAK}
\begin{center}
	\center
	\begin{singlespace}
		\large\bfseries\MakeUppercase{\thetitle}

		\normalfont\normalsize
		Oleh:

		\bfseries \theauthor
	\end{singlespace}
\end{center}

\begin{singlespace}
	\small
	Lorem ipsum dolor sit amet . Operator grafis dan tipografi mengetahui hal ini dengan baik, pada kenyataannya semua profesi yang berhubungan dengan alam semesta komunikasi memiliki hubungan yang stabil dengan kata-kata ini, tetapi apa itu? Lorem ipsum adalah teks dummy tanpa arti. Ini adalah urutan kata Latin yang, sebagaimana posisinya, tidak membentuk kalimat dengan pengertian yang utuh, tetapi memberikan kehidupan pada teks uji yang berguna untuk mengisi ruang-ruang yang selanjutnya akan ditempati dari teks ad hoc yang disusun oleh para profesional komunikasi.

	Ini tentu saja yang paling terkenal teks placeholder bahkan jika ada versi berbeda yang dapat dibedakan dari urutan pengulangan kata-kata Latin. Lorem ipsum berisi tipografi yang lebih banyak digunakan, sebuah aspek yang memungkinkan Anda untuk memiliki gambaran umum tentang rendering teks dalam hal pilihan font dan d ukuran font. Jika mengacu pada Lorem ipsum, ekspresi yang digunakan berbeda, yaitu fill text , fiktif text , blind text atau placeholder text : singkatnya, artinya juga bisa nol, tetapi kegunaannya sangat jelas untuk bertahan selama berabad-abad dan menolak versi ironis dan modern yang datang dengan kedatangan web.

	\textbf{\textit{Kata kunci: Lorem, Ipsum, Lorem Ipsum }}

\end{singlespace}
\clearpage
% \input{chapters/abstract-en}

\chapter*{Kata Pengantar}
\addcontentsline{toc}{chapter}{KATA PENGANTAR}

Puji dan syukur penulis panjatkan kepada Tuhan Yang Maha Esa atas berkat dan rahmatnya, laporan tugas akhir yang berjudul "\thetitle" dapat diselesaikan dalam rangka memenuhi syarat kelulusan tingkat sarjana. Perlu diakui pengerjaan tugas akhir ini didukung oleh banyak pihak. Khususnya, penulis ingin mengucapkan terima kasih kepada:

\begin{enumerate}
	\item Bapak Dr.techn. Muhammad Zuhri Catur Candra, S.T., M.T., selaku dosen pembimbing atas segala bentuk dukungan yang telah diberikan dan kesabarannya dalam membimbing penulis serta memberikan saran dalam pengerjaan tugas akhir.
	\item Bapak Yudistira Dwi Wardhana Asnar, S.T, Ph.D dan Dr. Agung Dewandaru, S.T., M.Sc., selaku dosen penguji atas segala masukan dan kritik yang telah diberikan terhadap tugas akhir penulis.
	\item Dicky Prima Satya, S.T, M.T., Bapak Adi Mulyanto, S.T, M.T., Robithoh Annur, S.T., M.Eng., Ph.D., dan Tricya Esterina Widagdo, ST., M.Sc. selaku dosen koordinator tim tugas akhir atas usahanya mengingatkan mahasiswa program studi Teknik Informatika untuk mengerjakan tugas akhirnya.
	\item Seluruh dosen program studi Teknik Informatika ITB yang telah memberikan ilmu pengetahuan yang sangat berharga bagi penulis.
	\item Ibu Rini Liani dan Bapak Edhie Hikmat selaku kedua orangtua penulis atas dukungan yang diberikan
	\item Rumah Amal Salman ITB serta tim Beasiswa Perintis yang telah membantu penulis sehingga penulis dapat menempuh pendidikan di Institut Teknologi Bandung dengan mudah.
	\item Teman-teman INIT 2020 yang telah menemani, memberikan inspirasi, serta dukungan moral kepada penulis dalam menempuh kuliah pada program studi Teknik Informatika.
	\item Teman-teman penulis khususnya anggota dari grup "temenin ngerjain TA", "Koordinasi penonton sempro", "Para Ajudan Pecinta Sedekah", "Kos Aufa Enjoyer", "kaliMANTAN", serta "karimun" yang telah memberikan kenangan berharga, motivasi, hiburan, serta bantuan untuk segala situasi.
	\item Sahabat Penulis khususnya Erik dan Rachel yang telah pantang menyerah berjuang bersama dalam berkompetisi CTF.
	\item Sahabat terdekat penulis khususnya Marcho, Aira, Gagas, Rio, Dhika, Kinan, Anca, Dipa, Ubai, Azka, Sarah, Dea, Fay, Epi, Aufa, Syahrul, Rifqi dan Malik yang telah menemani perjuangan dari TPB hingga saat ini, menjadi \textit{emotional support} di segala situasi, membantu penulis dalam proses pengejaan tugas akhir, serta membuat hari - hari menjadi lebih berwarna.
	\item Seluruh pihak lain yang tidak bisa disebutkan disini yang telah membantu dalam proses pengerjaan tugas akhir.
\end{enumerate}

Akhir kata, penulis mengucapkan terima kasih kepada semua pihak yang telah terlibat dalam pengerjaan tugas akhir ini. Penulis juga ingin menyampaikan mohon maaf apabila terdapat kesalahan maupun kekurangan dalam laporan tugas akhir ini. Penulis berharap semoga tugas akhir ini dapat bermanfaat bagi pembaca dan riset-riset kedepannya.

\begin{flushright}
	\vspace{0.5cm}
	Bandung, \tanggalpengesahan

	\vspace{1.5cm}

	Muhammad Garebaldhie ER Rahman
\end{flushright}

\titleformat*{\section}{\centering\bfseries\Large\MakeUpperCase}
\titlespacing*{\chapter}{0pt}{0pt}{4pt}

% Setting judul toc, lot, lof, bib
\renewcommand{\contentsname}{DAFTAR ISI}
\renewcommand{\listfigurename}{DAFTAR GAMBAR}
\renewcommand{\listtablename}{DAFTAR TABEL}
\renewcommand{\bibname}{DAFTAR PUSTAKA}

% daftar isi, lampiran, gambar, table
\tableofcontents
\listofappendices
\listoffigures
\listoftables

\newpage

\titleformat*{\section}{\bfseries\large}
\pagenumbering{arabic}

%----------------------------------------------------------------%
% Konfigurasi Bab
%----------------------------------------------------------------%
\setcounter{page}{1}
\renewcommand{\chaptername}{BAB}
\renewcommand{\thechapter}{\Roman{chapter}}
%----------------------------------------------------------------%

%----------------------------------------------------------------%
% Dafter Bab
% Untuk menambahkan daftar bab, buat berkas bab misalnya `chapter-6` di direktori `chapters`, dan masukkan ke sini.
%----------------------------------------------------------------%
\input{chapters/ta/chapter-1}
\chapter{Studi Literatur}
\label{chapter:studi-literatur}

Bab Studi Literatur digunakan untuk mendeskripsikan kajian literatur yang terkait dengan persoalan tugas akhir. Tujuan studi literatur adalah:

\begin{enumerate}
	\item menunjukkan kepada pembaca adanya gap seperti pada rumusan masalah yang memang belum terselesaikan,
	\item memberikan pemahaman secukupnya kepada pembaca tentang teori atau pekerjaan terkait yang terkait langsung dengan penyelesaian persoalan, serta
	\item menyampaikan informasi apa saja yang sudah ditulis/dilaporkan oleh pihak lain (peneliti/Tugas Akhir/Tesis) tentang hasil penelitian/pekerjaan mereka yang sama atau mirip kaitannya dengan persoalan tugas akhir.
\end{enumerate}

Kita juga dapat memisah beberapa bagian latex untuk \textit{readability}. Kita dapat memasukan file latex lainnya dengan menggunakan fitur input. Berikut merupakan contoh ketika memasukan bagian contoh-subbab ke chapter-2

\section{Contoh Subbab}
Perujukan literatur dapat dilakukan dengan menambahkan entri baru di berkas. Tulisan ini merujuk pada \parencite{vasp1} atau \parencite{codingame2021media} dan \parencite{4026885}

Sekarang mau ke bab berapa yaaaa.... hmm... ke bab \ref{chapter:studi-literatur} ahhhhh.

\blindtext

\subsection{Contoh Subsubbab}
\label{subsec:contoh-subsec}

\blindtext

\begin{figure}[ht]
	\centering
	\includegraphics[width=0.5\textwidth]{resources/cover-ganesha.jpg}
	\caption{Contoh gambar}
\end{figure}

\subsubsection{Subsubsubbab}
\blindtext

\subsubsubsection{sub sub sub sub bab}
\blindtext

\begin{table}[h]
	\caption{Tabel random}
	\vspace{0.25cm}
	\begin{center}
		\begin{tabular}{|c|c|c|c|}
			\hline
			Title1 & Title2 & Title3 & Title4  \tabularnewline
			\hline
			1647   & 1.97   & 0.68   & 1.90 \tabularnewline
			2301   & 2.92   & 1.06   & 2.75 \tabularnewline
			2969   & 3.23   & 1.16   & 3.78 \tabularnewline
			3791   & 4.39   & 1.40   & 4.14 \tabularnewline
			4625   & 6.72   & 1.87   & 5.59 \tabularnewline
			\hline
		\end{tabular}
	\end{center}
\end{table}

\section{Menyisipkan Persamaan}

Beberapa contoh menyisipkan persamaan.

\subsection{Contoh Bikin Equation}
\textbf{text tebal} dan ini \emph{miring}, bikin persamaan di baris yang sama, tinggal pake dolar2 $\Psi(\vec{r}_1,...,\vec{r}_N)$, sehingga persamaan Schr\"{o}dinger, terus, persamaan yang dinomeri kayak gini 
%ini contoh bikin persamaan, ..... :D
\begin{equation}
	\left[ \sum_{i}^{N}-\frac{\hbar^2}{2m}\nabla_i^2 + \sum_{i}^{N}V(\vec{r}_i)+ \sum_{i<j}^{N}(\vec{r}_i,\vec{r}_j)\right]\Psi = E\Psi 
\end{equation}

untuk $N$-elektron, dengan $\hat{H}$=Hamiltonian, $E$=Energi total, $\hat{T}$=Energi kinetik, $\hat{V}$=Energi potensial, dan $\hat{U}$=Interaksi ektron-elektron.

\subsection{Bikin Matrix}
Lalalallala.... bikin matrix sekarang, yang ini dikecilin, pake smaller
	{\smaller
		\begin{equation}
			\Psi({\bf r}_1, {\bf r}_2, \cdots {\bf r}_N) = \frac{1}{\sqrt{N!}}\left| \begin{array}{llcl}
				\phi_1({\bf r}_1)     & \phi_2({\bf r}_1)     & \cdots                & \phi_N({\bf r}_1)     \\
				\phi_1({\bf r}_2)     & \phi_2({\bf r}_2)     & \cdots                & \phi_N({\bf r}_2)     \\
				\phi_1({\bf r}_3)     & \phi_2({\bf r}_3)     & \cdots                & \phi_N({\bf r}_3)     \\
				\multicolumn{1}{c}{.} & \multicolumn{1}{c}{.} & \multicolumn{1}{c}{.} & \multicolumn{1}{c}{.} \\
				\multicolumn{1}{c}{.} & \multicolumn{1}{c}{.} & \multicolumn{1}{c}{.} & \multicolumn{1}{c}{.} \\
				\multicolumn{1}{c}{.} & \multicolumn{1}{c}{.} & \multicolumn{1}{c}{.} & \multicolumn{1}{c}{.} \\
				\phi_1({\bf r}_N)     & \phi_2({\bf r}_N)     & \cdots                & \phi_N({\bf r}_N)     \\
			\end{array} \right|
		\end{equation}
	}
% \blankpage
\chapter{Analisis Persoalan dan Rancangan Solusi}

Tujuan utama penulisan bab ini adalah untuk menguraikan rencana penyelesaian masalah implementasi dari judul TA

\section{Analisis}
\blindtext

\section{Rancangan}
\blindtext

\chapter{Implementasi dan Pengujian}
Bab ini akan menjelaskan proses implementasi dari rancangan solusi yang telah dikaji pada Bab III. Setelah pembahasan terkait implementasi, akan dilanjutkan dengan pemaparan hasil uji terkait implementasi yang telah dibuat.

\section{Lingkungan}
\blindtext

\section{Implementasi}
\blindtext

\section{Pengujian}
\blindtext



\input{chapters/ta/chapter-5}
%---------------------------------------------------------------%

% Daftar pustaka
\printbibliography

% Setting judul lampiran
\titlespacing*{\chapter}{0pt}{0pt}{0pt}
\titlespacing*{\section}{0pt}{0pt}{*1}

% Setting judul anak lampiran
\titleformat*{\section}{\bfseries}

\appendix

\chapter{Contoh gambar pengujian}

\begin{figure}[ht]
	\centering
	\includegraphics[width=0.5\textwidth]{resources/cover-ganesha.jpg}
	\caption{Contoh gambar}
\end{figure}


\end{document}
