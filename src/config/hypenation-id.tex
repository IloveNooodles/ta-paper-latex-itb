%--------------------------------------------------------------------%
%
% Hypenation untuk Bahasa Indonesia
%
% @author Petra Barus
%
%--------------------------------------------------------------------%
%
% Secara otomatis LaTeX dapat langsung memenggal kata dalam dokumen,
% tapi sering kali terdapat kesalahan dalam pemenggalan kata. Untuk
% memperbaiki kesalahan pemenggalan kata tertentu, cara pemenggalan
% kata tersebut dapat ditambahkan pada dokumen ini. Pemenggalan
% dilakukan dengan menambahkan karakter '-' pada suku kata yang
% perlu dipisahkan.
%
% Contoh pemenggalan kata 'analisa' dilakukan dengan 'a-na-li-sa'
%
%--------------------------------------------------------------------%

\hyphenation {
  % A
  %
  a-na-li-sa
  a-pli-ka-si
  a-lo-ka-si
  an-ta-ra
  ada-nya
  akan
  % B
  %
  be-be-ra-pa
  ber-ge-rak
  be-ri-kut
  ber-ko-mu-ni-ka-si
  buah
  Bouvet
  ber-fo-kus
  ber-fung-si
  ber-ja-lan
  % C
  %
  ca-ri
  con-strained
  Carzaniga
  cloud
  CloudFormation
  con-tai-ner
  ClusterIP
  % D
  %
  da-e-rah
  di-nya-ta-kan
  de-fi-ni-si
  di-bu-tuh-kan
  di-gu-na-kan
  di-tam-bah-kan-nya
  di-tem-pat-kan
  di-la-ku-kan
  de-ploy-ment
  di-kem-bang-kan
  di-im-ple-men-ta-si-kan
  de-activation
  da-pat
  di-ka-te-go-ri-kan
  de-ngan
  di-kem-bang-kan
  da-ta
  % E
  %
  e-ner-gi
  eks-klu-sif
  eks-ter-nal
  % F
  %
  fa-si-li-tas
  % G
  %
  ga-bung-an
  % H
  %
  ha-lang-an
  ha-sil
  hell
  % I
  % 
  i-nduk
  in-for-ma-si
  im-ple-men-tasi
  % J
  %
  % K
  %
  kom-po-si-si
  ka-re-na
  ke-sa-ba-ran-nya
  ka-me-ra
  kua-li-tas
  ke-na-ngan
  kom-plek-si-tas
  ke-ti-ka
  ke-le-bi-han-nya
  ke-gi-a-tan
  ko-mu-ni-ka-si
  ke-cil
  % L
  %
  la-ya-nan
  % M
  %
  me-ngu-ra-ngi
  meng-eva-lu-a-si
  me-nge-lo-la
  men-da-lam
  men-ja-lan-kan
  mak-si-mal
  me-nye-le-sai-kan
  me-ngunjungi
  men-du-kung
  me-nu-rut
  me-la-ku-kan
  mem-buat
  men-daftar-kan
  me-ngu-sul-kan
  me-mi-li-ki
  meng-gu-na-kan
  men-ja-di
  me-ru-pa-kan
  men-ja-ga
  me-mu-dah-kan
  me-ne-rus-
  mem-pro-ses
  % N
  %
  Na-mun
  % O
  %
  ob-so-lete
  or-kes-tra-si
  oto-ma-ti-sa-si
  % P
  %
  pro-vi-der
  pe-ru-sa-ha-an
  pe-rang-kat
  pro-ses
  plat-form
  pro-duk-si
  pe-ne-li-tian
  pe-ru-ba-han
  pa-ra-dig-ma
  pe-man-tau-an
  pe-ngum-pu-lan
  pa-ckage
  % Q
  %
  quality
  % R
  %
  re-source
  re-mote
  % S
  se-la-in
  stan-dar-di-sasi
  se-cu-ri-ty
  so-lu-si
  se-lu-ruh
  Soft-ware
  soft-ware
  se-buah
  se-ca-ra
  %
  % T
  % 
  ter-li-bat
  ter-pi-sah-kan
  ter-mi-nal
  ter-ba-tas
  % U
  %
  un-tuk
  % V
  %
  % W
  %
  % X
  %
  % Y
  % 
  % Z
  %
  zoo-keeper
}
